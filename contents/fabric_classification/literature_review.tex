\section{Literature Review}

Fabric classification has been an area of growing interest in the fields of computer vision and material science due to its wide range of applications across industries such as fashion, healthcare, manufacturing, and forensics. With the advancement of deep learning, researchers have been exploring automated methods to accurately identify and classify textile materials using image data.

Traditionally, fabric classification relied on manual techniques and physical testing, but these approaches are time-consuming, prone to human error, and often require specialized equipment. In contrast, image-based classification using deep learning models offers a scalable and efficient alternative that can be used even in real-time applications.

In this section, we review key research contributions that have applied deep learning techniques for fabric classification. The datasets described here, used in these studies are carefully considered, as data quality and diversity are vital to achieving strong model performance. We also provide a detailed discussion of the three papers that were selected and implemented as part of this research work. Each paper highlights a different modeling approach offering valuable insights into the strengths and limitations of current methodologies.

By analyzing these works, we aim to identify the common practices, innovative ideas, and existing challenges in the domain, which in turn have shaped the direction of our proposed methodology.