\chapter{Fabric Classification using Deep Learning}

Textiles can be found globally across many industries—not only apparel and fashion, but also in home furnishings, healthcare, automotive and aerospace industries, etc. Each fabric exhibits a unique texture, appearance, and structure. There are over 25,000 fabric types, and successfully identifying the fabrics can be challenging. Fabric identification is important because the fabric type typically dictates what the fabric is suitable for e.g., garments (apparel), medical (medical supplies), and interior (furnishings). 

In the past, fabrics were exclusively identified by manual and chemical means, which included burn tests, solubility tests, and microscopic examination of fibers. Although the manual and chemical methods can be useful, they can be time-consuming and require proficient practitioners. With industries that require quick decisions or involve large-scale processing, such as fast fashion or textile manufacturing, the manual identification approaches are not always suitable.

Recent advances in computer vision and deep learning open up new possibilities for fabric classification automation, and deep learning models can learn fine-grained patterns and textures through images, therefore enabling reliable model fabric classification without handcrafted features; additionally, vision transformers (ViTs) have become robust models able to model global relationships in images, therefore ViT models may be able to model complex texture and structure for fabric. 

In this chapter, we compare deep learning methods of classifying fabrics with images. The first method is based on transfer learning using a VGG-16 model from a set of fabric images. In the second method, TextileNet, we compare multiple pre-trained CNN models from two datasets: Optical Coherence Tomography (OCT) fabric images and macro fabric images. We have illustrated the comparably strong performance of MobileNetV2 when working with the OCT images and the third method is using a Vision Transformer as a feature extractor combined with PCA, LDA and a support vector machine (SVM) for classification scores.

By utilizing and analyzing these methods, we come to understand both the benefits and the limitations of the two methods of classification. Based on the insights of the two methods of classification, we propose a new model that combines a CNN branch and a Vision Transformer branch to improve the robustness and accuracy of fabric classification.

% Overview of the research project.