\chapter{Introduction}

As a part of my Master’s program, I had the opportunity to work on two very different but equally valuable projects—one in research and the other in a practical, industry-based environment. This thesis brings together both experiences in a single document. The first part describes a research project I worked on about the topic Fabric classification using deep learning, and the second part focuses on my internship work in the field of Security Orchestration, Automation, and Response (SOAR).

Both projects helped me to step out of my comfort zone and exposed me to different aspects of the tech world. The research work taught me how to approach technical problems from a scientific perspective that included reading papers, building models, analyzing results—all while diving deep into machine learning. On the other hand, my internship was much more hands-on, where I got a change to be part of a team developing a cybersecurity tool. This involved writing code for both the backend and frontend, working with APIs, and figuring out how everything ties together in a real world software system.

These experiences were not just about learning new tools or technologies but they also gave me a clear understanding of how AI is being used to solve real problems, whether it’s about predicting threat responses or identifying fabric types using images. I also learned a lot from the people I worked with, who shared their own knowledge and helped me think more critically and practically about building solutions.

\section{Research Work - Fabric Classification using Deep Learning}
% Briefly introduce your research work here.

My research project focused on \textbf{fabric classification using deep learning}, a critical application within computer vision and artificial intelligence. The objective was to develop an automated system capable of accurately identifying different types of fabrics from images. This problem holds significant relevance in various industries, including textile manufacturing, fashion retail, and quality control, where manual classification is often time-consuming, prone to human error, and lacks scalability.

My work involved a comprehensive exploration of deep learning techniques, specifically \textbf{convolutional neural networks (CNNs)} and \textbf{Vision Transformers}, which are exceptionally well-suited for image-based recognition tasks. The research process encompassed several key stages:
\begin{itemize}
    \item \textbf{Literature Review and Implementation}: I began by delving into existing research, specifically implementing and analyzing three prominent papers in the field:
    \begin{enumerate}
        \item ``\textbf{Research on Classification of Clothing Fabrics Images Based on Convolutional Neural Network}'' This paper provided foundational insights into using CNNs for fabric classification.
        \item ``\textbf{TextileNet: A Deep Learning Approach for Textile Fabric Material Identification from OCT and Macro Images}'' This work expanded my understanding of employing deep learning for textile material identification across different image modalities.
        \item ``\textbf{Fabric Composition Identification using Fine-Tuned Vision Transformers}'' This research introduced me to the advanced capabilities of Vision Transformers in fine-grained image classification tasks, which proved invaluable for identifying fabric compositions.
    \end{enumerate}
    \item \textbf{Data Collection and Preprocessing}: Following the insights gained from these implementations, I curated and prepared a diverse dataset of fabric images, addressing challenges such as varying lighting conditions, textures, and image resolutions to ensure robust model training.
    \item \textbf{Model Development and Training}: Beyond replicating existing models, I then proceeded to build and implement my own deep learning model for fabric classification. This involved designing a custom architecture, experimenting with different configurations, and performing extensive hyperparameter tuning to optimize performance.
    \item \textbf{Performance Evaluation}: I rigorously assessed both the implemented models from the literature and my custom-built model's accuracy, precision, recall, and F1-score to determine their effectiveness in classifying various fabric types. A comparative analysis highlighted the strengths and weaknesses of each approach, demonstrating the efficacy of my proposed solution.
\end{itemize}
This research not only provided practical insights into applying deep learning for image classification but also significantly enhanced my understanding of scientific methodology, including comprehensive literature review, experimental design, model development, and critical analysis of results. The findings from this project culminated in a publication, demonstrating its contribution to the academic discourse.

\section{BEL Internship - Security, Orchestration, Automation and Response}
My internship at Bharat Electronics Limited (BEL) provided an invaluable opportunity to engage with the practical aspects of cybersecurity, specifically in the domain of \textbf{Security, Orchestration, Automation, and Response (SOAR)}. In today's complex threat landscape, Security Operations Centers (SOCs) face an overwhelming volume of alerts and incidents, making manual response impractical and inefficient. SOAR platforms are designed to address these challenges by integrating security tools, automating routine tasks, and orchestrating complex incident response workflows.

During my internship, I was an integral part of a team developing a SOAR solution aimed at enhancing the efficiency and effectiveness of threat detection and response within a real-world operational environment. My contributions involved:
\begin{itemize}
    \item \textbf{Backend Development}: Implementing robust backend logic to process security alerts, integrate with various security tools via APIs, and manage incident data. This included working with databases and designing efficient data pipelines.
    \item \textbf{Frontend Development}: Creation of a user-friendly web interface for the SOAR platform, enabling security analysts to visualize alerts, manage incidents, and execute automated response actions.
    \item \textbf{AI Model Integration}: Developing a machine learning model capable of understanding natural-language descriptions of incidents and recommending action to mitigate the incident. This included dataset preparation, model training, evaluation, and integration with the SOAR backend.
\end{itemize}
This hands-on experience provided me with a deep understanding of the practical challenges in cybersecurity operations and the critical role of automation in mitigating risks. It also honed my skills in full-stack development, collaborative software engineering, and problem-solving within a dynamic industry setting. The successful completion of this internship is marked by an official certificate from Bharat Electronics Limited.