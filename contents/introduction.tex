\chapter{Introduction}

As a part of my Master’s program, I got the change to work on two very different but equally valuable projects—one in research and the other in a practical, industry-based environment. This thesis brings together both experiences in a single document. The first part is about a research project on the topic Fabric classification using deep learning, and the second part is about internship work in the field of Security Orchestration, Automation, and Response (SOAR).

Both the projects gave me a chance to explore different aspects of the tech world. The research work taught me about how to approach technical problems from a scientific point of view. It includes reading papers, building models and analyzing results for different experiment in the field of Machine Learning. On the other hand, my internship was much more hands-on, where I got a change to be part of a team developing a cybersecurity tool. This internship taught me about building the backend and frontend, working with APIs, and how everything works together in a real software.

Both of these experiences not just helped me to learn new tools or technologies but they also gave me a chance to get clear understanding of how AI is used to solve real problems. I also learned a lot from the people who worked with me. They shared their knowledge and also helped me to think in a more logical and practical way to build a software.

\section{Research Work - Fabric Classification using Deep Learning}
% Briefly introduce your research work here.

The research work is about the topic "Fabric Classification using Deep Learning" which is a relevant  in the field of computer vision and artificial intelligence. The objective of this project work was to develop a machine learning model that is capable to detect and classify different types of fabrics with the help of given fabric image. This problem is relevant in various industries such as textile manufacturing, fashion, clothing, home furnishing, furniture, etc where fabrics are detected manually and the classification of fabric could be time-consuming and have high chances of having a human error in the process.

In my work, I experimented with various deep learning approaches, specifically convolutional neural networks (CNNs) and Vision Transformers. Both of these approaches have great applications for recognition tasks that include images. There are several stages of research that need to be considered in general such as:
\begin{itemize}
    \item \textbf{Literature Review and Implementation}: I started with a literature review to identify research that aligned to my study, specifically I implemented and reviewed three popular research articles relevant to this work:
    \begin{enumerate}
        \item ``\textbf{Research on Classification of Clothing Fabrics Images Based on Convolutional Neural Network}'' This paper provided a basic understanding of the application of CNNs for classifying fabrics. 
        \item ``\textbf{TextileNet: A Deep Learning Approach for Textile Fabric Material Identification from OCT and Macro Images}'' This paper extended my understanding of using deep learning related to identifying textile materials from two types of Fabrics dataset.
        \item ``\textbf{Fabric Composition Identification using Fine-Tuned Vision Transformers}'' This research directed me to understand the capabilities of Vision Transformers for fine-grained image classification, which has been beneficial for identifying fabric compositions.
    \end{enumerate}
    \item \textbf{Data Collection and Preprocessing}: After lessons learnt from these implementations, I collected and processed a dataset that contains images with conditions such as different light, textural differences and image resolution to allow effective training of my model.
    \item \textbf{Model Development and Training}: After my previous implementations and using them as a reference point, I went on to build and implement my own deep learning model for fabric classification. I developed a new and independent architecture, with various configurations and hyperparameters to maximize performance.
    \item \textbf{Performance Evaluation}: I evaluated the accuracy, precision, recall, and F1-score of both the implemented models of the literature, and my model developed independently, to investigate the usefulness of each approaches to classify multiple classes of fabric. A comparative analysis of each model describes how and where they succeed and fail, showcasing the strength of my proposed solution.
\end{itemize}
The research provided me with not only practical experience in implementing deep learning for image classification, but substantially improved my understanding of scientific practice, including literature review, experimental design, model development, and evaluated results.

\section{BEL Internship - Security, Orchestration, Automation and Response}
My internship experience at \textbf{Bharat Electronics Limited (BEL)}exposed me to real-world applications of cybersecurity, specifically \textbf{Security Orchestration, Automation and Response (SOAR)}. With today's varied threat landscape, Security Operations Centers (SOCs) have to process millions of alerts and incidents, making it impractical or impossible to respond manually. SOAR introduced a new level of integration across security tools, automation of manual tasks, and orchestration of complex incident response workflows. My internship consisted of work to build a SOAR solution to improve threat detection and response processes in a functional operational environment. My responsibilities included:

\begin{itemize}
    \item \textbf{Backend Development}: Develop solid backend logic to ingest the security alerts, aggregate relevant data through APIs from multiple security tools, and then maintain the incident data.  It involved the database work and developing system to pass the data to UI.
    \item \textbf{Frontend Development}: Development of a user-friendly web interface for SOAR platform to allow security analysts to see alerts, manage incidents and perform automated response actions related to incidents with the help of workflows that can be made using the UI.
    \item \textbf{AI Model Integration}: Building a machine learning model that provides recommendations for actions that will assist with incident mitigation,by taking natural language descriptions of incidents as input. This included dataset preperation, model training, and integration with the SOAR backend.
\end{itemize}
Overall, going through this real world learning opportunity provided insightful perspective regarding the real world, commercial challenges in cybersecurity operations, and how automation \& decision support can help to mitigate risk. It further allowed me to develop my experience and skills, such as those required in full-stack software development, commercial software engineering team, and problem solveing within a fast changing sector. 