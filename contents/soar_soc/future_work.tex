\section{Future Work}

While the developed SOAR platform fulfills many fundamental requirements of modern Security Operations Centers (SOCs), there are countless possibilities for further improvements, enhancements, and expansion. Future direction can only expand the capabilities of the platform, applicability to more complex environments, and resilience in operation.

\begin{itemize}
    \item \textbf{Role-Based Access Control (RBAC):} The existing software currently offers basic user authentication. Future iterations could extend granular RBAC to allow more access control permissions for users depending on their user role (i.e., analyst, incident responder, administrator) optimizing security a level of accountability in operation.
    
    \item \textbf{Multi-Tenant Support:} To enable use for multiple organizational units or departments, multi-tenancy could be added to the platform, to include isolated data storage, tenant-specific playbooks, and access controls.

    \item \textbf{Real-Time Threat Intelligence Feeds:} As the platform supports static integrations for the time being, the dynamic ingesting of real-time threat intelligence from commercial and open-source feeds (e.g. MISP, STIX/TAXII) would add depth to incident enrichment, situational awareness, and response accuracy.

    \item \textbf{Machine Learning Integration:} The actions for the incidents that are received from the SIEM platform can be enhanced with machine learning algorithms to predict the required actions based on historical data. This would reduce the manual effort of the analysts required to find out the action that is required to mitigate the incident.

    \item \textbf{Integration with SIEM and EDR Platforms:} While the platform is intentionally architected to have modular connections, this solution even benefits tremendously from native ingest from common platforms such as Splunk, ELK, Sentinel, or CrowdStrike Falcon, which would create a complete end-to-end response pipeline.

    \item \textbf{SLA and Case Management Enhancements:} Adding capabilities to support Service-Level Agreements (SLA), task assignments, case priorities, and escalation rules will allow the platform to be viable with enterprise-level incident management practices. 

    \item \textbf{Audit Logging and Compliance Reporting:}  Audit trails for each action taken in the system and compliance reporting (e.g., NIST, ISO 27001), and additional logging, reporting, and accountability will make this platform compliant with regulated industries (e.g., finance, healthcare, and defense). 

    \item \textbf{Distributed Deployment and High Availability:} Distributed deployments with a focus on either scalability, operational continuity with support for containerization, orchestration (e.g., Kubernetes), and fault-tolerant architecture.

    \item \textbf{Integration with MITRE D3FEND:} Currently the platform connects with MITRE ATT\&CK framework; however, the addition of MITRE D3FEND would give the option of depicting detection/movement versus defensive countermeasures, with additional value to SOC data-driven discussions.

\end{itemize}

These future enhancements reflect where the industry is headed and address operational requirements in critical infrastructure and defense environments which have high expectations for reliability, flexibility, and context aware response. Ongoing research and development in these areas would position the platform as a full-spectrum, resilient solution for next generation cybersecurity operations.
