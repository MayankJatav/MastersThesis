\section{Security Operations Center (SOC)}\vspace{-0.5em}

A \textbf{Security Operations Center (SOC)} is a centralized organizational unit responsible for continuously monitoring, detecting, analyzing, and responding to security incidents. Its ultimate goal is to protect the confidentiality, integrity, and availability of organizational information assets. The SOC is the first line of defense in an enterprise cybersecurity strategy and operates 24/7 to detect and respond to threats as they occur. Driven by cloud-based adoption, remote workforces, and transformation in business practices, the need for an effective SOC is necessarily critical, and many organizations need help validating the effectiveness and efficiency of their SOCs. An SOC can leverage existing IT, and business reporting structures for efficient triage and want to drive and continuously improve efficiency and escalating investigative capabilities. As IBM characterizes it, the SOC combines people, processes, and technology in delivering a centralized and coordinated security threat response across the entire digital infrastructure of an organization~\cite{ibm}.

\subsection{Key Components of a SOC}\vspace{-0.5em}

A SOC depends on the interdependent functions of its three fundamental components - people, processes, and technology. The people element consists of a team of cybersecurity professionals operating at several tiers of responsibilities. For example, Tier 1 analysts are generally responsible for alert triage when they arrive at the SOC, minimising false positives and given a priority to alerts. Tier 2 and Tier 3 analysts further investigate current incident handling, threat hunting, and forensics capabilities. SOC managers have overall control of the team, defined adherence to operational standards, and generate reporting and escalation. Microsoft states that the most effective SOC operations depend heavily on clearly defined roles, and structured willingness to collaborate~\cite{microsoft}. 

The second pillar, processes, includes documented workflows, escalation protocols and response protocols that are in alignment with common international standards, such as ISO/IEC 27035, and the NIST 800-61 incident handling guide. The documented processes create consistency in how an organization manages incidents and allows for regulatory compliance in incident management. 
The third component, technology, enables all of the SOC functions. This includes a variety of tools; e.g., SIEM - security information and event management, EDR - endpoint detection and response, NDR - network detection and response, firewall, IDS/IPS - intrusion detection/prevention systems, and threat intel platform (TIP). All these systems work together to provide visibility, aggregating data, alerting, and context. As pointed by Splunk, integrations and interoperability of these tools is important to reduce noise, higher detection being and seamless incident workflows~\cite{splunk}.

\subsection{Core Functions of a SOC}\vspace{-0.5em}

The role of a SOC is to maintain 24/7 observance over an organization’s digital landscape while responding to cybersecurity events. The SOC continuously watches the network traffic, endpoint logs, application behavior, and user actions to uncover abnormalities and indicators of compromise. When a potential threat is identified, the SOC initiates its incident response process, which consists of using the incident for threat validation, root cause analysis, containment, eradication, and recovery. The SOC uses post incident reviews to adjust its detection logic, update and refine its response playbooks, and avoid repeating mistakes. SOC teams are involved in proactive activities as well, including threat hunting, where analysts try to find unidentified threats based on behavioral hypotheses, and vulnerability management to minimize attack surfaces. Compliance enforcement is also a fundamental SOC responsibility in many sectors that have regulatory frameworks (e.g., HIPAA, PCI DSS, GDPR, and various countries' data protection laws). According to Gartner, high-functioning SOCs do not only respond to known threats, they also anticipate potential emerging risks and are programmed to implement controls prior to an incident occurring~\cite{gartner}. Additionally, the SOC is responsible for formulating and articulating security metrics (e.g., Mean Time to Detect (MTTD), Mean Time to Respond (MTTR)), to promote executive awareness of the effectiveness of cybersecurity operations.

\subsection{Types of SOCs}\vspace{-0.5em}

The establishment and implementation of a SOC can be very different based on the organization size, resources, and risk tolerance.
\begin{itemize}[noitemsep, topsep=0pt]
    \item A Dedicated SOC is a completely in-house capability where the organization owns all the security infrastructure and personnel. This model provides an organization with total control of their security and provides an optimal experience for large companies and critical government capabilities.
    \item A Virtual SOC (vSOC) is inexpensive, decentralized, and often located in a SaaS cloud with teams connecting for monitoring at different locations using digital collaboration tools for remote monitoring of security with potentially limited control and visibility.
    \item A Managed SOC contract (which is a Managed Security Services Provider MSSP), when a organizations lack internal expertise and goes to third-party security monitoring and incident response (pay to get plan for external support). This is sensible for small to mid-sized organizations as well.
    \item Finally, the Hybrid SOC capability combines internal security operations and external service providers toward the triple goals of cost-effectiveness, control, and coverage through incident response.
\end{itemize}
MITRE emphasizes that an organization must evaluate their risk tolerance level, regulatory requirements, and state of operational maturity in order to make a good decision on SOC operational models~\cite{mitre_soc}.

\subsection{SOC Maturity and Challenges}

SOC maturity refers to the organization's capability to detect, analyze, and respond to threats efficiently and proactively. Gartner’s SOC Maturity Model categorizes SOCs into three broad stages: reactive, proactive, and predictive. Reactive SOCs rely on basic alerting and manual processes; proactive SOCs use structured playbooks, threat intelligence, and begin to implement automation; predictive SOCs incorporate machine learning, user behavior analytics, and big data to anticipate attacks before they occur. However, achieving high maturity is a significant challenge. Most organizations struggle with \textit{alert fatigue}—a condition where analysts are overwhelmed by the volume of alerts, many of which are false positives. This often leads to delayed responses and burnout. Another major challenge is \textit{tool sprawl}—the presence of too many disparate tools that are poorly integrated, resulting in data silos and inefficient workflows. The \textit{cybersecurity skills shortage} further compounds these issues, with many SOCs facing difficulty in hiring and retaining qualified personnel. MITRE highlights that without strategic investment in training, tool consolidation, and process automation, SOCs remain trapped in reactive cycles, unable to scale against modern threat landscapes~\cite{mitre}.

\subsection{Role of SOAR in the SOC}

To address the growing demands on SOCs, many organizations are adopting \textbf{Security Orchestration, Automation, and Response (SOAR)} platforms as a key enhancement. SOAR platforms integrate with existing SOC tools and enable automation of routine and repetitive tasks such as alert enrichment, IP blocking, and log collection. According to Palo Alto Networks, SOAR empowers security teams by standardizing workflows, orchestrating cross-tool communication, and reducing analyst fatigue through automation~\cite{paloalto}. This leads to faster incident response, improved accuracy, and reduced Mean Time to Detect (MTTD) and Mean Time to Respond (MTTR). SOAR also introduces structured playbooks that define how different types of incidents should be handled, reducing human error and ensuring consistent actions across the SOC. In high-stakes environments, such as national defense and critical infrastructure sectors, the use of SOAR enhances the SOC’s ability to respond to advanced persistent threats and ensures operational continuity.
