\chapter{Security Orchestration, Automation, and Response (SOAR)}

Security Orchestration, Automation, and Response (SOAR) platforms are rapidly integrating into SOC architecture, enabling teams to automate repetitive tasks, allow tools to work together seamlessly with automation, and minimize incident response time. Each component of the acronym indicates a fundamental function that has collectively provided modern SOCs the capability of operating at greater speed, consistency and, intelligence. The components of SOAR include: 

\begin{itemize}
    \item \textbf{Security} \\
    Indicates cybersecurity which is the protection of systems, networks, and data from cyber threats. In the context of SOAR, Security is the area of application and where the focus is protecting an organization from threats such as malware, phishing, data breaches, etc.

    \item \textbf{Orchestration} \\
    Orchestration is the combination and coordination of different security tools and technologies. A SOAR platform can connect with SIEMs, firewalls, endpoint detection and response (EDR), threat intelligence data feeds, ticketing systems, etc. using APIs, enabling them to work together in one integrated incident response workflow.

    \begin{quote}
        \textit{Example:} For example, SOAR system receives an alert from SIEM platform, it gets IP reputation information from a threat intelligence platform, block the IP using a firewall, and create a helpdesk ticket.
    \end{quote}

    \item \textbf{Automation} \\
    Automation refers to the performance of predetermined tasks without human intervention. Such tasks could include alert triage, collection of logs, enrichment of IP/domain/file indicators, and threat classification. Automation greatly reduces the burden on SOC analysts and gives them more time to respond to incidents in a timely manner.

    \item \textbf{Response} \\
    Response is the final step with the SOAR life cycle, where actions are taken to remediate or contain the incident. Depending on the set rules within SOAR, the response could be:
    \begin{itemize}[noitemsep, topsep=0pt]
        \item \textit{Fully automated} – e.g. block a malicious IP.
        \item \textit{Semi-automated} – e.g. isolate a device with analyst approval.
        \item \textit{Manual} – the SOAR platform will assist the analyst by enriching any data and providing context.
    \end{itemize}
    The response phase is designed to help the analyst consistently, document, and manage incidents for auditing.
\end{itemize}

Collectively, they form the basis of a modern, intelligent, and scalable incident management system for any sufficiently sophisticated SOC.

Traditional manual processes for threat detection and mitigation are becoming less effective in light of the rapidly expanding scale and sophistication of cyber threats. The modern enterprise network, which often involves multiple cloud services, remote users, and many connected digital endpoints, cannot afford to keep using manual processes, and requires a more rapid and structured incident management approach. In this environment, Security Orchestration, Automation, and Response (SOAR) platforms have become a revolutionary solution to more effectively enhance the capabilities of security operation centers (SOCs) through integration, automation, and intelligence-driven response.

SOAR platforms are a response to the challenges presented to organizations by a combination of their internal security ecosystem and related personnel, and their incident response processes. The common sources of inefficiency included disparate systems (i.e. security information and event management [SIEM] platforms; firewalls; endpoint detection and response [EDR] tools; and threat intelligence services) and the lack of an overall, coordinated response to security incidents. Moreover, as defined by Palo Alto Networks, SOAR platforms coordinate, execute, and automate tasks between various people and tools from within a single platform to enhance the efficiency and reliability of security function and corresponding processes.~\cite{paloalto}.

This chapter gives a detailed overview of SOAR, based on my experiences completing an internship at Bharat Electronics Limited (BEL) - an exceptional defense public sector company that has done advanced work throughout the defense sector. I completed this internship mainly to build an understanding and develop a design and framework for a custom SOAR platform that is designed for both enterprise and defense cases. Although I cannot reveal implementation specifics and technical solutions, the architecture and research I performed lay the basis for the discussion in the chapter. 

SOAR platforms consist of several tightly integrated modules: 
\begin{itemize}[noitemsep,topsep=0pt]
    \item \textbf{Incident Ingestion} using a SIEM platform (e.g., IBM QRadar, Microsoft Sentinel) that generates actionable alerts based on log correlation~\cite{microsoftsiem}.
    \item \textbf{Tool Orchestration} modules which enabled heterogeneous tools to communicate using common APIs and custom integration~\cite{techtarget}.
    \item \textbf{Automation Workflows and Playbooks} which enabled predesigned tasks like threat enrichment, IP block, and priority alerts to run autonomously, without human interaction~\cite{paloalto}.
    \item \textbf{Response Interfaces} that allow analysts to manually interact, inspect, investigate, and take action when automation cannot be relied upon or may be risky~\cite{techtarget}.
    \item \textbf{MITRE ATT\&CK Framework Integration} which allows standardized mapping of tactics and techniques used by attackers that inform operational plans to defensively guide actions in response to threats, as well as track threat movements~\cite{mitre}.
\end{itemize}

These abilities cut down on manual work and analyst fatigue and also increase response time and consistency across the entire organization’s security ecosystem. Overall, they lead to positive changes in Mean Time to Detect (MTTD) and Mean Time to Respond (MTTR), both key metrics within today’s SOC environment.

The remaining of this chapter will cover the primary functional components of a SOAR platform and how orchestration, automation, and structured response can help to enhance cyber defenses.