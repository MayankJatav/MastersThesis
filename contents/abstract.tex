\chapter*{Abstract}

This thesis presents a dual-contribution in the domains of computer vision and cybersecurity by addressing two distinct yet impactful challenges: fabric classification and security operations automation. The first part of the research focuses on developing a deep learning-based framework for the classification of textile fabrics. A hybrid architecture combining Convolutional Neural Networks (CNN) and Vision Transformers (ViT) is proposed to effectively capture texture pattern in fabrics for classification. The model is trained and evaluated on a dataset comprising cotton, polyester, and wool fabrics, achieving superior accuracy compared to state-of-the-art models. This work contributes toward automated material recognition for industrial applications such as textile quality control and intelligent sorting.

The second part of the thesis is devoted to the design and development of a modular Security Orchestration, Automation, and Response (SOAR) platform. Developed as part of an internship at Bharat Electronics Limited, the platform features secure user authentication, real-time dashboard analytics, incident management with AI-assisted mitigation, visual workflow automation, MITRE ATT\&CK integration, and dynamic app-based response actions. Compared to commercial and open-source SOAR solutions, the custom platform provides a lightweight, extensible, and context-aware alternative suitable for deployment in sensitive environments such as defense and critical infrastructure.

Together, the two contributions demonstrate a multidisciplinary application of artificial intelligence and system design methodologies across material science and cybersecurity. The outcomes highlight the practical impact of integrating deep learning models with real-world industrial tasks, and the importance of automation platforms in improving incident response workflows in modern Security Operations Centers.
