\chapter{Conclusion}

This thesis presented a dual-focused exploration into two contemporary technological domains: deep learning for material classification and security automation in cyber defense. The two distinct yet impactful components—\textbf{Fabric Classification using Deep Learning} and \textbf{Security Orchestration, Automation, and Response (SOAR)}—were investigated, implemented, and evaluated with the objective of addressing real-world challenges in their respective fields.

The first part of the thesis concentrated on developing a robust deep learning framework for the classification of textile fabrics. Leveraging both macro-level RGB images and micro-level Optical Coherence Tomography (OCT) data, a hybrid model integrating Convolutional Neural Networks (CNN) and Vision Transformers (ViT) was proposed. This approach enabled the effective extraction of both local texture features and global contextual representations, resulting in significant improvements in classification accuracy across multiple fabric types such as cotton, polyester, and wool. Comparative evaluations against state-of-the-art models—including TextileNet and fine-tuned ViT architectures—demonstrated the effectiveness of the proposed methodology. The work not only contributes to advancements in automated material recognition but also opens avenues for deployment in industries such as textile quality control, recycling, and inventory management.

The second part of the thesis focused on the design and development of a modular, AI-enhanced SOAR platform as part of an internship at Bharat Electronics Limited (BEL). The platform was designed to automate and orchestrate cybersecurity incident detection and response workflows within a SOC environment. Key components of the system include secure authentication, real-time dashboard analytics, incident management with AI-based mitigation, modular integrations, dynamic playbook and workflow execution, and an interactive MITRE ATT\&CK mapping module. The system was benchmarked against leading commercial platforms such as Splunk SOAR and Cortex XSOAR, as well as open-source tools like Shuffle. The developed platform distinguishes itself through its tight integration of ATT\&CK techniques, support for visual workflow construction, and the inclusion of machine learning model selection for intelligent response decisions. It is especially suited for deployment in sensitive and restricted environments such as defense and critical infrastructure.

Collectively, this thesis demonstrates a strong synthesis of theory, applied research, and practical system development across two independent domains. The first part contributes to the field of computer vision and deep learning, while the second addresses critical needs in cybersecurity automation. Both systems are modular, extensible, and designed with future scalability in mind.

Future work in fabric classification could involve expanding the dataset to include blended fabrics, incorporating multispectral imaging, or deploying the model in real-time quality inspection pipelines. For the SOAR platform, enhancements could include role-based access control (RBAC), integration with real-time threat intelligence feeds, and advanced anomaly detection using reinforcement learning.

In conclusion, this thesis exemplifies a multidisciplinary approach to solving complex, real-world problems using modern AI and system design methodologies. It contributes novel insights and working prototypes that have the potential for real-world application in both industrial and security-critical contexts.
