\chapter{Conclusion}

This thesis includes the presentation of two prominent technologies: deep learning for materials classification; and security automation in cyber defense. Two separate but meaningful components namely, Fabric Classification using Deep Learning and Security Orchestration, Automation and Response (SOAR), were investigated, implemented, and evaluated with the intent of solving real-world applications in their individual fields.

The initial part of the thesis was to develop a comprehensive deep learning framework to classify textile fabrics. Using macro-level RGB images and micro-level Optical Coherence Tomography (OCT) data, we proposed a hybrid model of Convolutional Neural Networks (CNN) to extract the local texture features, and Vision Transformers (ViT), to extract a higher-level contextual representation of the image of the fabric. The use of the hybrid approach yielded significant increases in classification accuracy across multiple types of fabrics (cotton, polyester, and wool). The constructed baseline model was contrasted against leading-edge models such as TextileNet as well as fine-tuned ViT architectures, both of which further confirmed the effectiveness of the proposed methodology. Besides general improvements to automate material recognition in the field, the contribution can be positioned for use in applications such as textile quality control, textile recycling and textile inventory control.

The second phase of the thesis covered the development and design of a modular, AI-enabled SOAR platform as an intern at Bharat Electronics Limited (BEL). The SOAR platform automates and orchestrates workflows for cybersecurity incident detection and response in a SOC environment. The SOAR is composed of secure authentication, dashboards in real time analytics, incident management that incorporates some AI-based mitigation, modular integration, dynamic playbook and workflows, and an interactive MITRE ATT\&CK mapping module. The system was benchmarked against leading commercial platforms such as Splunk SOAR and Cortex XSOAR, as well as open-source tools like Shuffle. The developed platform distinguishes itself through its tight integration of ATT\&CK techniques, support for visual workflow construction, and the inclusion of machine learning model selection for intelligent response decisions. It is especially suited for deployment in sensitive and restricted environments such as defense and critical infrastructure.

Collectively, this thesis demonstrates a strong synthesis of theory, applied research, and practical system development across two independent domains. The first part contributes to the field of computer vision and deep learning, while the second addresses critical needs in cybersecurity automation. Both systems are modular, extensible, and designed with future scalability in mind.

Future work on fabric classification might include adding blended fabrics classification datasets, using multispectral imagers, or placing the model in a real-time quality inspection pipeline. For the SOAR platform, next steps would include adding a role-based access control (RBAC) system, integrating real-time threat intelligence feeds, and developing more advanced anomaly detection algorithms by using reinforcement learning.

To conclude, this thesis demonstrates a multidisciplinary approach to solving complex real-world problems using modern AI and systems design approaches. It introduces illustrated perspectives and working prototypes, that can potentially have real-world impact in both industrial and security-critical environments.